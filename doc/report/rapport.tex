\documentclass[12pt]{report}

\usepackage[francais]{babel}
\usepackage[utf8]{inputenc}
\usepackage[T1]{fontenc}
\usepackage{amsmath}
\usepackage{amssymb}
\usepackage[cyr]{aeguill}
\usepackage{fancyheadings}
\usepackage[pdftex]{graphicx}
\DeclareGraphicsExtensions{.jpg,.pdf,.png}
\usepackage[pdftex,colorlinks=true,linkcolor=blue,citecolor=blue,urlcolor=blue]{hyperref}
\usepackage{anysize}
\usepackage{verbatim}
\marginsize{22mm}{14mm}{12mm}{25mm}
\usepackage{natbib}
\usepackage{icomma}
\setlength{\parskip}{.3cm}

\begin{document}
\pagestyle{fancyplain}
\renewcommand{\chaptermark}[1]{\markboth{\chaptername\ \thechapter. #1}{}}
\renewcommand{\sectionmark}[1]{\markright{\thesection. #1}}
\lhead[]{\fancyplain{}{\bfseries\leftmark}}
\rhead[]{\fancyplain{}{\bfseries\thepage}}
\cfoot{}

\makeatletter
\def\figurename{{\protect\sc \protect\small\bfseries Fig.}}
\def\f@ffrench{\protect\figurename\space{\protect\small\bf \thefigure}\space}
\let\fnum@figure\f@ffrench%
\let\captionORI\caption
\def\caption#1{\captionORI{\rm\small #1}}
\makeatother
\edef\hc{\string:}
\graphicspath{{img/}}

%%%%%%%%%%%%%%%%%%%%%%%%%%%%%%%%%%%%%%%%%%%%%%%%%%%%%%%%%% Couverture :
\thispagestyle{empty}
{\Large
\begin{center}
Prénom NOM
\vskip1cm

%% Pour redéfinir la distance entre la boite et le texte
\fboxsep6mm
%% Pour redéfinir l'épaisseur de la boite
\fboxrule1.3pt

%% Le \vphantom{\int_\int} sert à introduire de l'espace entre les deux lignes
%% (essayez donc de le commenter)
$$\fbox{$
  \begin{array}{c}
  \textbf{Titre}
  \vphantom{\int_\int}
  \end{array}
  $}
$$
\end{center}
\vskip8cm

\begin{flushright}
\textit{Encadrant :}

Zacharie ALES
\end{flushright}
}

\clearpage

%%%%%%%%%%%%%%%%%%%%%%%%%%%%%%%%%%%%%%%%%%%%%%%%%%%%%%%%%% Table des matières :
\renewcommand{\baselinestretch}{1.30}\small \normalsize

\tableofcontents

\renewcommand{\baselinestretch}{1.18}\small \normalsize


%%%%%%%%%%%%%%%%%%%%%%%%%%%%%%%%%%%%%%%%%%%%%%%%%%%%%%%%%% Introduction :
\chapter{Introduction}

\section{Exemples}

\subsection{Bases}

% L'environnement minipage permet de mettre  côte à côte deux zones de texte
% - [t] indique que les zones sont alignées en haut (top)
% - .45\linewidth indique que la largeur des zones de texte est égale à 45% de la zone de texte de la page
\begin{minipage}[t]{.45\linewidth}
\textbf{Rendu dans le pdf}
\vspace{.5cm}

Text normal.

\textit{Texte italique.}

\textbf{Texte en gras.}

\underline{Texte souligné.}


\end{minipage}\hfill\vrule\hfill
\begin{minipage}[t]{.45\linewidth}
\textbf{Code correspondant en latex}

\begin{verbatim}
Text normal.
\textit{Texte italique.}
\textbf{Texte en gras.}
\underline{Texte souligné.}

\end{verbatim}
\end{minipage}
\subsection{Listes}


\begin{minipage}[t]{.45\linewidth}
\textbf{Rendu dans le pdf}
\vspace{.5cm}

Liste sans numéro :
\begin{itemize}
\item item 1 ;
\item item 2.
\end{itemize}
Liste avec numéro :
\begin{enumerate}
\item item 1 ;
\item item 2.
\end{enumerate}

\end{minipage}\hfill\vrule\hfill
\begin{minipage}[t]{.45\linewidth}
\textbf{Code correspondant en latex}

\begin{verbatim}
Liste sans numéro :
\begin{itemize}
\item item 1 ;
\item item 2.
\end{itemize}
Liste avec numéro :
\begin{enumerate}
\item item 1 ;
\item item 2.
\end{enumerate}
\end{verbatim}
\end{minipage}

\subsection{Formules mathématiques}

% L'environnement minipage permet de mettre côte à côte deux zones de texte
\begin{minipage}[t]{.45\linewidth}
\textbf{Rendu dans le pdf}
\vspace{.5cm}

  $ \alpha_j  = \varepsilon_1 +  z^2 + \frac  {1 -  \delta}{2 +
    \gamma} + \sum_{i=1}^n w_i\quad \forall j\in\{1, ..., p\}$
\end{minipage}\hfill\vrule\hfill
\begin{minipage}[t]{.45\linewidth}
\textbf{Code correspondant en latex}

\begin{verbatim}
$\alpha_j = \varepsilon_1 + 
 z^2 + \frac {1 - \delta}{2 +
 \gamma} + \sum_{i=1}^n w_i
 \quad \forall j\in\{1, ..., p\}$
\end{verbatim}
\end{minipage}


\textit{Remarques} :
\begin{itemize}
\item Le texte en indice ou en exposant doit être entouré
  d'accolades sauf s'il ne contient qu'un unique caractère ;
\item \textbackslash quad permet d'espacer des éléments dans une formule.
\end{itemize}
\vspace{.3cm}

\begin{minipage}[t]{.45\linewidth}
\textbf{Rendu dans le pdf}
\vspace{.5cm}

Exemple d'équation numérotée :

\begin{equation}
  x = y
  \label{eq:monEquation}
\end{equation}

Référence à cette équation : \eqref{eq:monEquation}.
\end{minipage}\hfill\vrule\hfill
\begin{minipage}[t]{.45\linewidth}
\textbf{Code correspondant en latex}

\begin{verbatim}
Exemple d'équation numérotée :

\begin{equation}
  x = y
  \label{eq:monEquation}
\end{equation}

Référence à cette équation : 
\eqref{eq:monEquation}.
\end{verbatim}

\end{minipage}


\subsection{Tableaux}

\begin{minipage}[t]{.45\linewidth}
\textbf{Rendu dans le pdf}
\vspace{.5cm}

\begin{tabular}{lcr}
  \hline
  \textbf{Titre 1} 
  & \textbf{Titre 2} 
  & \textbf{Titre 3} \\

  \hline
  c1 & c2 & c3 \\

  c4 & c5 & c6\\

  \hline
\end{tabular}
\end{minipage}\hfill\vrule\hfill
\begin{minipage}[t]{.45\linewidth}
\textbf{Code correspondant en latex}

\begin{verbatim}
\begin{tabular}{lcr}
  \hline
  \textbf{Titre 1} 
  & \textbf{Titre 2} 
  & \textbf{Titre 3} \\

  \hline
  c1 & c2 & c3 \\

  c4 & c5 & c6\\

  \hline
\end{tabular}
\end{verbatim}
\end{minipage}






\textit{Remarques sur ce tableau :}
\begin{itemize}
\item Le tableau contient trois colonnes :
  \begin{itemize}
  \item la première est centrée à gauche('l' : left) ;
  \item la seconde est centrée ('c' : center) ;
  \item la troisième est centrée à droite ('r' : right)
  \end{itemize}
\item ``\textbackslash hline'' représente une ligne horizontale. Cette
  instruction doit toujours \^etre située au début d'une ligne ;
\item '\&' indique la fin d'une case ;
\item ``\textbackslash\textbackslash'' représente la fin d'une ligne.
\end{itemize}
\vspace{.3cm}


Il  est  généralement  préférable  de mettre  les  tableaux  dans  des
environnement  tables  qui  sont  numérotés et  peuvent  contenir  une
légende. C'est le cas de la table~\ref{tab:ex}.


\begin{table}[h!]
\begin{minipage}[t]{.45\linewidth}
\textbf{Rendu dans le pdf}
\vspace{.5cm}

  \centering

  \begin{tabular}{lcr}
    \hline
    \textbf{Titre 1} 
    & \textbf{Titre 2} 
    & \textbf{Titre 3} \\

    \hline
    c1 & c2 & c3 \\

    c4 & c5 & c6\\

    \hline
  \end{tabular}
\end{minipage}\hfill\vrule\hfill
\begin{minipage}[t]{.45\linewidth}
\textbf{Code correspondant en latex}

\begin{verbatim}
\begin{table}[h!]

  \centering
  % ... contenu de la table

  \caption{Exemple de table.}
  \label{tab:ex}
\end{table}
\end{verbatim}
\end{minipage}

  \caption{Exemple de table.}
  \label{tab:ex}
\end{table}

\textit{Remarques sur la table~\ref{tab:ex}} :
\begin{itemize}
 \item Le placement des tables et figures est géré par latex.  On
  ne choisit pas où elles seront situées par rapport au texte.
\item L'option ``[h!]'' permet de demander à latex de mettre la figure
  dès que possible (sinon il aura  tendance à le mettre au début d'une
  page suivante).
\end{itemize}


\subsection{Problème d'optimisation}

\begin{minipage}[t]{.4\linewidth}
\textbf{Rendu dans le pdf}
  \begin{center}
$(P)\left\{
  \begin{array}{rll}
    \min_{x}& \sum_{i=1}^n w_i x_i\\

    \mbox{s.c.} & p_{i,j} x_i \leq B 
    & \forall j\in\{1, ..., m\}\\

    & x_i\in\mathbb N
  \end{array}
\right.$
            \end{center}
          \end{minipage}\hfill\vrule\hfill
\begin{minipage}[t]{.5\linewidth}
  \begin{center}
\textbf{Code correspondant en latex}
\begin{verbatim}
$(P)\left\{
  \begin{array}{rll}
    \min_{x}& \sum_{i=1}^n w_i x_i\\

    \mbox{s.c.} & p_{i,j} x_i \leq B 
    & \forall j\in\{1, ..., m\}\\

    & x_i\in\mathbb N
  \end{array}
\right.$
\end{verbatim}

            \end{center}
          \end{minipage}


\textit{Remarques :}
\begin{itemize}
\item   "\textbackslash  left"   et   "\textbackslash  right"   permettent
  d'encadrer un
  tableau par  un symbole (ici  une accolade à  gauche "\textbackslash
  left$\{$" et rien à droite "\textbackslash right.") ;
\item  l'environnement array  est similaire  à tabular  mais son
  contenu est au format mathématique ;
\item "\textbackslash mbox" permet d'écrire du texte dans une formule mathématique.
\end{itemize}


\subsection{Figures}

La figure~\ref{fig:maFigure} représente le logo de l'ENSTA.

\begin{figure}[h!]

\centering
\begin{minipage}[t]{.3\linewidth}
\textbf{Rendu dans le pdf}
\vspace{.5cm}

\centering  \includegraphics[height=4cm]{Logo_ENSTA_Paris.png}
\end{minipage}\hfill\vrule\hfill
\begin{minipage}[t]{.6\linewidth}
\textbf{Code correspondant en latex}

\begin{verbatim}
\begin{figure}[h!]
  \centering
  \includegraphics[height=4cm]{Logo_ENSTA_Paris.png}
  \caption{Légende}
  \label{fig:maFigure}
\end{figure}
\end{verbatim}
\end{minipage}
  \caption{Légende}
  \label{fig:maFigure}
\end{figure}

\textit{Remarque :}
\begin{itemize}
\item La commande  \textbackslash graphicspath$\{\{$img$/\}\}$, utilisée
  au début du document, indique à  latex d'y chercher les images. C'est
  pour cela qu'il est suffisant d'indiquer le nom de l'image (pas son répertoire).
\end{itemize}

\subsection{Astuces}

Si vous  ne savez pas comment  écrire un symbol en  latex, vous pouvez
le dessiner sur le site
\href{http://detexify.kirelabs.org/classify.html}{http://detexify.kirelabs.org/classify.html}
qui vous indiquera la commande correspondante.



\subsection{Références bibliographiques}
 
Pour citer un article en latex il faut :
\begin{enumerate}
\item  Trouver  le  fichier  .bib  associé  à  cet  article  (on  peut
  généralement le trouve sur le site google
  scholar). Exemple :
\begin{verbatim}
@article{bertsimas2017optimal,
  title={Optimal classification trees},
  author={Bertsimas, Dimitris and Dunn, Jack},
  journal={Machine Learning},
  volume={106},
  number={7},
  pages={1039--1082},
  year={2017},
  publisher={Springer}
}
\end{verbatim}

;
\item   Copier   le   contenu   de  ce   fichier   dans   le   fichier
  "bibliography.bib" ;
\item Citer ce fichier dans votre rapport en utilisant son identifiant (dans l'exemple
  l'identifiant   est    bertsimas2017optimal)   dans    la   commande
  ``\textbackslash cite''. Voici la référence de l'article précédent : \cite{bertsimas2017optimal}
\end{enumerate}


\addcontentsline{toc}{chapter}{Bibliographie}

%% Feuille de style bibliographique : monjfm.bst
\bibliographystyle{apalike}
\bibliography{bibliography}

\end{document}


\begin{minipage}[t]{.45\linewidth}
\textbf{Rendu dans le pdf}
\vspace{.5cm}

\end{minipage}\hfill\vrule\hfill
\begin{minipage}[t]{.45\linewidth}
\textbf{Code correspondant en latex}

\begin{verbatim}
\end{verbatim}
\end{minipage}